\documentclass[dvipsnames,table]{beamer}
\usepackage{polski}

\usetheme{Rochester}
\usecolortheme{orchid}

\usepackage{listings}
\usepackage{ucs}
\usepackage[utf8x]{inputenc}
\usepackage{wasysym}
\usepackage[normalem]{ulem}
\usepackage{amsmath}
\usepackage{hyperref}
\usepackage{tikzsymbols}

\setbeamertemplate{navigation symbols}{}
\setbeamertemplate{caption}[numbered]
\setbeamerfont{caption}{size=\scriptsize}
\setbeamercolor{framenote}{bg=OSEC-red!25}
\setbeamercolor{rednote}{bg=Red!25}
\setbeamercolor{palette primary}{use=structure,fg=white,bg=OSEC-red}
\setbeamercolor{palette secondary}{use=structure,fg=white,bg=OSEC-red2}

\setbeamertemplate{itemize item}{\scriptsize\raise1pt\hbox{\donotcoloroutermaths$\blacktriangleright$}}
\setbeamertemplate{itemize subitem}{\tiny\raise1pt\hbox{\donotcoloroutermaths$\bullet$}}
\setbeamertemplate{itemize subsubitem}{\tiny\raise1pt\hbox{\donotcoloroutermaths{--}}}

\setbeamertemplate{enumerate item}{\insertenumlabel.}
\setbeamertemplate{enumerate subitem}{\insertenumlabel.\insertsubenumlabel}
\setbeamertemplate{enumerate subsubitem}{\insertenumlabel.\insertsubenumlabel.\insertsubsubenumlabel}
\setbeamertemplate{enumerate mini template}{\insertenumlabel}

\setbeamercolor{itemize item}{fg=OSEC-red, bg=OSEC-red}
\setbeamercolor{itemize subitem}{fg=OSEC-red, bg=OSEC-red}
\setbeamercolor{itemize subsubitem}{fg=OSEC-red, bg=OSEC-red}

\setbeamercolor{section number projected}{fg=white,bg=OSEC-red}
\setbeamercolor{subsection number projected}{fg=white,bg=OSEC-red}
\setbeamercolor{button}{bg=OSEC-red,fg=white}

\setbeamertemplate{section in toc}[circle]
\setbeamertemplate{subsection in toc}[square]

\definecolor{OSEC-red}{RGB}{160,29,44}
\definecolor{OSEC-red2}{RGB}{177,76,12}
\hypersetup{colorlinks=true,linkcolor=white,urlcolor=OSEC-red}

\setlength{\tabcolsep}{8pt}
\renewcommand{\arraystretch}{1.2}

\newcommand{\tri}{$\triangleright$ }


\newcommand\YAMLcolonstyle{\color{red}\mdseries}
\newcommand\YAMLkeystyle{\color{black}\bfseries}
\newcommand\YAMLvaluestyle{\color{black}\mdseries}

\makeatletter

\newcommand\language@yaml{yaml}

\expandafter\expandafter\expandafter\lstdefinelanguage
\expandafter{\language@yaml}
{
  keywords={true,false,null,y,n},
  keywordstyle=\color{darkgray}\bfseries,
  basicstyle=\YAMLkeystyle,                                 % assuming a key comes first
  sensitive=false,
  comment=[l]{\#},
  morecomment=[s]{/*}{*/},
  commentstyle=\color{purple}\ttfamily,
  stringstyle=\YAMLvaluestyle\ttfamily,
  moredelim=[l][\color{orange}]{\&},
  moredelim=[l][\color{magenta}]{*},
  moredelim=**[il][\YAMLcolonstyle{:}\YAMLvaluestyle]{:},   % switch to value style at :
  morestring=[b]',
  morestring=[b]",
  literate =    {---}{{\ProcessThreeDashes}}3
                {>}{{\textcolor{red}\textgreater}}1     
                {|}{{\textcolor{red}\textbar}}1 
                {\ -\ }{{\mdseries\ -\ }}3,
}

% switch to key style at EOL
\lst@AddToHook{EveryLine}{\ifx\lst@language\language@yaml\YAMLkeystyle\fi}
\makeatother

\newcommand\ProcessThreeDashes{\llap{\color{cyan}\mdseries-{-}-}}

\lstdefinestyle{yaml}
{
   language=yaml,
   basicstyle=\small\ttfamily,
   breaklines=true,
%   escapechar=\@,
}
\lstdefinestyle{bash}
{
   language=bash,
   basicstyle=\tiny\ttfamily,
   breaklines=true,
%   escapechar=\@,
}
\lstdefinestyle{cfg}
{
   basicstyle=\small\ttfamily,
   breaklines=true,
}

\title{Konteneryzacja w Red Hat Enterprise Linux 8}
\author{Radosław Kujawa -- radoslaw.kujawa@osec.pl}
\institute{OSEC}

\begin{document}

\begin{frame}
	\titlepage
\end{frame}

\begin{frame}
\frametitle{RHEL 8 a kontenery}
\begin{center}
\includegraphics[scale=0.1]{img-rhlogo.png}
\end{center}
\begin{itemize}
	\item 8 Maja 2019 -- Red Hat Enterprise Linux 8
\end{itemize}
\begin{center}
\end{center}
\end{frame}

\begin{frame}
	\frametitle{Brak Dockera w RHEL 8}
	\begin{itemize}
		\item Odpowiedź środowiska: projekt container tools \url{https://github.com/containers} 
	\end{itemize}
\end{frame}

\begin{frame}
\frametitle{YUM v4 (dnf)}
\begin{itemize}
	\item microdnf -- reimplementacja {\tt dnf} o małych wymaganiach
%	\item Modularyzacja systemu
	\item {\em Application Streams} -- odpowiedź na potrzeby twórców aplikacji, aby dostarczać wspierane alternatywne wersje oprogramowania
	\item Ułatwia budowę kontenerów z konkretną wersją oprogramowania
	\item Pojawią się strumienie z nowszymi wersjami paczek w ramach życia RHEL 8 (np. php)
	\item Budowanie paczek zgodnie z ideologią SCL w dalszym ciągu obsługiwane
\end{itemize}
\begin{center}
\end{center}
\end{frame}

\begin{frame}
\frametitle{YUM v4 (dnf)}
\begin{itemize}
	\item Usługi dostępne w alternatywnych wersjach
	\item Np. PostgreSQL 9.6 oraz 10
	\item Np. MariaDB 10.3 oraz MySQL 8
\end{itemize}
\end{frame}

\begin{frame}
\frametitle{YUM v4 (dnf)}
\begin{itemize}
	\item {\tt dnf module enable postgresql:10}
	\item {\tt dnf module install postgresql:10}
\end{itemize}
%\lstinputlisting[style=yaml]{../ansible/deploy-modular-postgresql.yml}
\end{frame}

\begin{frame}
	\frametitle{Nowe narzędzia do konteneryzacji}
\begin{itemize}
	\item RHEL 8 nie dostarcza Dockera
	\item Container tools -- {\tt podman}, {\tt buildah}, {\tt skopeo}
	\item Silnik dla instalacji lokalnych, developmentu: podman
	\item Silnik dla dużej skali (Kubernetes, OpenShift): \href{https://cri-o.io/}{CRI-O}
	\item {\tt udica} -- generator polityk SELinuxa dla kontenerów (RHEL 8.1)
% ipvlan w kontenerach
\end{itemize}
\end{frame}

\begin{frame}
	\frametitle{UBI}
	\begin{itemize}
		\item Foo
	\end{itemize}
\end{frame}

\begin{frame}
	\frametitle{podman}
	\begin{itemize}
		\item Nowoczesne narzędzie do zarządznia lokalnymi kontenerami
		\item W pełni wspierany, to nie jest {\em tech preview}!
		\item Bardzo intensywny rozwój
		\item Prostota konfiguracji, brak daemonów
		\item Kontenery działające na kontach użytkowników nieuprzywilejowanych
		\item Dobra integracja z systemd
		\item Ułatwienie migracji do Kubernetesa ({\tt podman generate kube, podman play kube})
	\end{itemize}
\end{frame}

\begin{frame}[fragile]
	\frametitle{{\tt podman} -- integracja z {\tt systemd}}
	\begin{itemize}
		\item Kontenery {\tt podman} mogą działać jako usługi {\tt systemd}
		\item Także na koncie nieuprzywilejowanego użytkownika
		\item {\tt ExecStart} może uruchamiać pojedyńczy kontener lub wykonywać {\tt play kube}
		\item Mechanizm {\em socket activation} w połączeniu z kontenerami
	\end{itemize}
%\begin{lstlisting}[style=cfg]
%\lstinputlisting[style=cfg]{../podman-systemd/mycontainer.service}
% or stunnelized stdio
\end{frame}

\begin{frame}
	\frametitle{Kompatybilność wsteczna z Docker}
	\begin{itemize}
		\item Paczka {\tt podman-docker} dostarcza polecenie {\tt docker}
		\item Implementacja Open Container Initiative (OCI) Runtime Specification -- obrazy zbudowane w RHEL container tools działają z Dockerem i na odwrót
%\footnote{Zazwyczaj\ldots}
		\item {\tt podman} nie stara się implementować API Dockera - jest oparty o własną bibliotekę {\tt libpod}
	\end{itemize}
\centering
	\begin{table}
\caption{Docker vs podman compatibility vs container tools native.}
\label{porownanie}
\scriptsize
\begin{tabular}{llll}
\hline
docker & rhel 8 compat & rhel 8 native   \\ \hline
	docker run & podman run & podman run \\
	docker build & podman build & buildah  \\
	docker pull & podman pull & skopeo copy \\ 
	\ldots & \ldots & \ldots \\ \hline
\end{tabular}
\normalsize
\end{table}
\end{frame}

\begin{frame}[fragile]
	\frametitle{Instalacja narzędzi do konteneryzacji za pomocą Ansible}
%	\lstinputlisting[style=yaml,lastline=11]{../ansible/deploy-container-tools.yml}
\end{frame}

\begin{frame}[fragile]
	\frametitle{Instalacja narzędzi do konteneryzacji za pomocą Ansible}
%	\lstinputlisting[style=yaml,firstline=12]{../ansible/deploy-container-tools.yml}
\end{frame}

\begin{frame}[fragile]
	\frametitle{Budowanie kontenera za pomocą {\tt buildah}}
%\lstinputlisting[style=bash]{../buildah-example/build-container.sh}
\begin{itemize}
	\item {\tt mkdir -p \$HOME/tmp/logs}
	\item {\tt sudo chcon -t container\_file\_t \$HOME/tmp/logs}
	\item {\tt podman run -p 7777:7777 -v \$HOME/tmp/logs/:/logs/ localhost/bcaas}
	\item {\tt -p} dla nieuprzywilejowanych kontenerów -- libpod issue \href{https://github.com/containers/libpod/issues/2081}{2081}
\end{itemize}
\end{frame}

\begin{frame}[fragile]
	\frametitle{Budowanie obrazu kontenera za pomocą Ansible}
%	\lstinputlisting[style=yaml]{../ansible/}
\begin{itemize}
	\item {\tt ansible-bender} vs {\tt ansible-container}?
	\item {\tt ansible-bedner} nie jest jeszcze dostępny jako paczka systemowa -- jest w {\tt pip} i na GitHub
	\item {\tt ansible-bender build container-provisioning.yml}
\end{itemize}
\end{frame}

\begin{frame}[fragile]
	\frametitle{Instalacja {\tt ansible-bender}}
\begin{center}
%\includegraphics[scale=0.13]{img-ansibleinception.jpg}
\end{center}
%\lstinputlisting[style=yaml,firstline=1,lastline=10]{../ansible/deploy-ansible-bender.yml}
\end{frame}

\begin{frame}[fragile]
	\frametitle{Instalacja {\tt ansible-bender}}
%\lstinputlisting[style=yaml,firstline=11]{../ansible/deploy-ansible-bender.yml}
\end{frame}

\begin{frame}[fragile]
	\frametitle{Tworzenie obrazów kontenerów z {\tt ansible-bender}}
%\lstinputlisting[style=yaml]{../ansible/container-provisioning.yml}
\end{frame}

\begin{frame}
\frametitle{Koniec\ldots}
\begin{center}
\href{https://github.com/OSEC-pl/barcamp-osec-rhel8containers}{https://github.com/OSEC-pl/barcamp-osec-rhel8containers}
\includegraphics[scale=0.5]{img-oseclogo.png}

Dziękuje!

Czy są pytania?

\end{center}
\end{frame}
\end{document}

